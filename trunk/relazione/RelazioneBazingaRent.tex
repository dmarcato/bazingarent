\documentclass[a4paper,12pt]{article}
\usepackage[italian]{babel}
\usepackage{graphicx}
\usepackage{lastpage}
\usepackage{fancyhdr}
\usepackage{color}
\usepackage{hyperref}
\pagestyle{fancy}


\title{\textbf{Relazione BazingaRent}}

\lfoot{BazingaRent}
\cfoot{}
\rfoot{\thepage /\pageref{LastPage}}

\renewcommand{\headrulewidth}{0.4pt}
\renewcommand{\footrulewidth}{0.4pt}

\renewcommand{\sectionmark}[1]{
\markboth{}{}}

\hypersetup{
colorlinks=true,
linkcolor=blue
}

\begin{document}
	 
	\maketitle
	
%%%%%%%%%%%%%%%%%%%%%%%%%%% SOMMARIO
	\begin{abstract}
	Questo documento contiene la relazione del progetto di Tecnologie Web denominato BazingaRent.
	\end{abstract}
	
%%%%%%%%%%%%%%%%%%%%%%%%%%% TABELLA INFO DOCUMENTO		
	\begin{center}
		\section*{}
		\begin{tabular}{r|l}
		
			\multicolumn{2}{c}{\bf Informazioni del documento}\\
			\hline
			{Redazione} & {A.Griggio}\\
			{Verifica}& {D.Calonego}\\
			{Approvazione}& {M.Broggio}\\
			{Stato}& Formale\\
			{Uso}& Esterno\\
			{Nome File}& nome\_file\\
			{Versione}& {1.0}\\
			{Distribuzione}& BazingaRent\\
			{}&{Gaggi Ombretta}\\
		\end{tabular}
	\end{center}	
	
	\newpage
	\tableofcontents
	\newpage
	
	%%%%%%%%%%%%%%%%%%%%%%%%%%%%% SEZIONE 1

	\section{Introduzione}
	BazingaRent ha lo scopo di rappresentare un sito internet per un negozio di video-noleggio.
	Nel sito si vogliono visualizzare: 
		\begin{itemize}
			\item {catalogo dove l'utente pu\`{o} cercare il/i film desiderato/i;}		
			\item {breve descrizione dei film;}			
			\item {disponibilit\`{a} dei film;}
			\item {ultimi arrivi;}
			\item {film pi\`{u} noleggiati;}
			\item {l'indirizzo della sede fisica del negozio con indicazioni per raggiungerlo, orari e contatti;}
			\item {film in arrivo, a breve disponibili.}
		\end{itemize}
	Considerando i contenuti del sito \`{e} stato ritenuto opportuno garantire l'accessibilit\`{a} alla maggior parte degli utenti, in particolar modo a quelli affetti da disturbi visivi. Inoltre lo stesso \`{e} stato realizzato in modo da poter essere visualizzato correttamente anche su dispositivi portatili, quali cellulari.
	Tutto questo rispettando gli standard Xhtml 1.0 Strict e Css 2.
	
	\newpage

	\section{Fase di progettazione}
	L'idea di base \`{e} stata quella di creare un sito semplice, intuitivo, ma allo stesso tempo gradevole e funzionale, con una struttura chiara e accessibile che si mantenesse tale in ogni pagina.
	Per garantire ci\`{o} il layout delle pagine \`{e} stato cos\`{i} pensato:
\\
\\
			\begin{center}
			\includegraphics{images/layoutpagina.png}
			\end{center}		

	\newpage

	Il men\`{u} di navigazione \`{e} stato pensato al di sotto dell'header con il nome e il logo del video-noleggio, \`{e} stato posizionato li sia per una questione di gusto estetico che per un esigenza pratica, cos\`{i} infatti viene lasciato pi\`{u} spazio alla visualizzazione dei film.
	\\
	Per quanto riguarda i film, si \`{e} pensato di disporli in appositi box a due a due all'interno di un contenitore principale.
	\\
	All'interno di ogni box sarebbero stati inseriti: 
	\begin{itemize}
	\item{l'immagine della locandina del film}
	\item{il titolo}
	\item{l'anno di uscita}
	\item{una breve descrizione della trama}
	\item{un link alla scheda completa del film}
	\end{itemize}
	Alla fine di ogni pagina \`{e} stato messo un footer contenente i loghi di validazione xhtml,css e il nome della compagnia.
	\\
	\\
	\\	
	Andremo ora ad analizzare la progettazione delle pagine pi� in dettaglio.

	\newpage
	\subsection{Home}
	In prima pagina abbiamo deciso di visualizzare gli ultimi arrivi, ovvero i film arrivati in data recente e da poco disponibili al noleggio.
	E' stato aggiunto prima dei box Film un breve testo di descrizione e benvenuto.	
	
	\subsection{I pi\`{u} noleggiati}
	Qui, come dice anche il nome, sono presenti il film maggiormente noleggiati dalla clientela.
	\\
	E' da considerarsi come una sorta di vetrina delle preferenze, dove l'utente dubbioso sulla scelta pu\`{o} rendersi conto di quali siano le pellicole pi\`{u} gettonate.
	
	\subsection{Prossimamente}
	In questa pagina invece sono contenuti i film non ancora disponibili, ma che presto arriveranno al noleggio.
	
	\subsection{Catalogo}
	Nel Catalogo sono esposti tutti i film disponibili presso il video-noleggio.
	\\
	Per questioni di ordine, abbiamo preferito suddividere i film per iniziale, inserendo prima dei box una {\sl lista} delle lettere in ordine alfabetico, ognuna quando selezionata permette di consultare l'insieme delle pellicole col titolo che ha quella specifica iniziale.
	E' stato inoltre integrato un form {\sl Cerca}, che permette, all'utente che magari ricorda solo parte del titolo, o a chi non volesse sfogliare i titoli per iniziale, di individuare facilmente il film desiderato.
	\\
	\\
	
	Le precedenti pagine condividono sostanzialmente la stessa struttura al fine di facilitare la navigazione all'utente medio.

	\subsection{Come trovarci}
	Questa pagina � l'unica che si differenzia dalle altre in quanto a struttura, e qui si possono ottenere le indicazioni per raggiungere la sede fisica del noleggio, sono presenti inoltre orari di apertura/chiusura, recapito telefonico e indirizzo e-mail.
	
	\newpage
	
	\subsection{Amministrazione}
	E' stata prevista anche una pagina di amministrazione, alla quale l'amministratore pu� accedere con login e password, e da cui pu� aggiungere, modificare, eliminare uno o pi� film.
	
	\newpage
	
	\section{Team di sviluppo}
	\begin{center}
		\begin{tabular}{r|l}
			\multicolumn{2}{c}{\bf BazingaSoft}\\
			\hline
			{Alberto Griggio} & {Sviluppatore xml, grafico, redattore}\\
			{Dario Marcato}& {Sviluppatore html e perl, esperto standard web}\\
			{Davide Calonego}& {Web designer, tester accessibilit�}\\
			{Marco Broggio}& {Project manager, sviluppatore perl e css}\\
		\end{tabular}
	\end{center}	
			
\end{document}